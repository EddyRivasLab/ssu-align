\subsection{Using ssu-build to create a Firmicutes-specific SSU model}
\label{sec:tutorial-build-firmicutes}

The default archaeal, bacterial and eukaryotic models provided with
SSU-ALIGN are meant to be general models that represent the SSU
sequence diversity in their respective domains. You might want to
create more specific models that cover a tighter or broader
phylogenetic range, for example a particular bacterial phyla of
interest, or a single model that covers all SSU sequences. The
\prog{ssu-build} program allows users to create such models. 

\prog{ssu-build} takes as input a multiple sequence alignment, called
a \emph{seed} alignment, of SSU sequences and creates a CM file that
includes a model that represents the diversity from the seed
alignment. This CM file can be used individually to align sequences,
or combined, through simple file concatenation, with other CM files
and used to classify and create clade-specific SSU alignments.

Typically, a seed alignment is carefully constructed to minimize
errors and ensure appropriate representation from the phylogenetic
clade being modeled. Errors in the seed alignment will lead to errors
in the output alignments from the CM built from that seed. Aligning
sequences to models that are outside the range of diversity in the
seed will also lead to errors. However, in this section, for
demonstration purposes, we'll build quick and dirty models from two
partitions of the default bacterial seed alignment included with
SSU-ALIGN. 

We'll create two bacterial SSU models, a
\emph{Firmicutes} specific model, and a model that represents all
other non-\emph{Firmicutes} bacteria. We'll then combine these two
models with the default archaeal and eukaryotic models and use them to
classify and align sequences from the \prog{seed-15.fa} sample dataset
used previously in this tutorial. 

\subsubsection{Partitioning the default bacterial seed alignment}

The first step is to identify which of the sequences in the default
bacteria seed are \emph{Firmicutes}. To do this, I ran the FASTA file
with the unaligned bacterial seed sequences in
\prog{seeds/bacteria-0p1.fa} through the RDP-CLASSIFIER tool
using default parameters \cite{Wang07} on the RDP website: \\
\htmladdnormallink{http://rdp.cme.msu.edu/classifier}{http://rdp.cme.msu.edu/classifier}\footnote{Version
  2.2 of RDP-CLASSIFIER was used with RDP training set 6.}
Seven of the 93 sequences are classified as \emph{Firmicutes} with an
80\% bootstrap threshold. These seven sequences are:
\begin{sreoutput}
01088::Bacillus_halodurans.::AB013373
01106::Bacillus_subtilis::K00637
01382::Clostridium_perfringens::M69264
01375::Clostridium_tetani::X74770|g509282
01295::Lactococcus_lactis_subsp._lactis.::AE006456
01252::Staphylococcus_aureus::L36472|g567883__bases_8528_to_10082_
01305::Streptococcus_pyogenes::AE006473
\end{sreoutput}

We'll use these sequences to build a \emph{Firmicutes} specific model. 
We'll use six of the seven sequences, omitting the \emph{Bacillus subtilis}
sequence for reasons that will become clear shortly.
%The other six are listed in the file \prog{firm-6.txt} in the
%\prog{ssu-align-0.1/tutorial/} directory. All seven are listed in the
%file \prog{firm-7.txt}. 
The remaining 86 seed sequences are non-\emph{Firmicutes} and we'll build 
a separate model from these. 

To build the models, we first need to extract the relevant aligned sequences
from the bacterial seed alignments using \prog{ssu-mask}. We'll use
the list files \prog{firm-7.list} and \prog{firm-6.list} in
\prog{ssu-align-0.1/tutorial/} to do this. The \prog{firm-7.list} file
includes the names of the seven \emph{Firmicutes} sequences, one per
line, and \prog{firm-6.list} includes the same names but without the
\emph{Bacillus subtilis} sequence. 

From the \prog{tutorial/} directory, we can create the two alignments
that we want with two commands:

\user{ssu-mask -a --seq-k firm-6.list --key-out firm-6 ../seeds/bacteria-0p1.stk}

\user{ssu-mask -a --seq-r firm-7.list --key-out nonfirm-86 ../seeds/bacteria-0p1.stk}

The first command creates the six sequence \emph{Firmicutes}-only
alignment in the file \\ \prog{bacteria-0p1.firm-6.seqk.stk} and the
second command creates the non-\emph{Firmicutes}
alignment in the file \prog{bacteria-0p1-nonfirm-86.seqr.stk}. 
In the first command, the \prog{--seq-k} option tells \prog{ssu-mask}
to keep only the six sequences listed in \prog{firm-6.list} In the
second command, the \prog{--seq-r} option tells \prog{ssu-mask} to
remove only the seven sequences listed in \prog{firm-7.list} and keep
all other sequences.  These commands only remove selected sequences
from the default seed in \prog{../seeds/bacterial-0p1.stk}, none of
the positions of the aligned nucleotides in the remaining sequences
have been changed.

Now, we can use \prog{ssu-build} to build models from the alignments
we've just created with two commands:

\user{ssu-build -n firmicutes -o firmicutes.cm bacteria-0p1.firm-6.seqk.stk}

\user{ssu-build -n bac-nonfirm -o bac-nonfirm.cm
  bacteria-0p1.nonfirm-86.seqr.stk}

The first command creates the file \prog{firmicutes.cm} which includes a CM
named \emph{firmicutes}, and the second command creates the file
\prog{bac-nonfirm.cm} which includes a CM named
\emph{bac-nonfirm}. 

We can use both of these models within \prog{ssu-align} simultaneously
to differentiate \emph{Firmicutes} SSU sequences from other bacterial
SSU sequences by concatenating them together:

\user{cat firmicutes.cm bac-nonfirm.cm > my.cm}

Since we'd also like to be able to differentiate archaeal and
eukaryotic sequences from bacterial sequences, we can also append
those default models:

\user{cat ../seeds/archaea-0p1.cm ../seeds/eukarya-0p1.cm >> my.cm}

Now the file \prog{my.cm} includes four CMs. Let's run
\prog{ssu-align} using this model on the \prog{seed-15.fa} dataset
we've been using throughout this tutorial:

\user{ssu-align -m my.cm seed-15.fa myseqs2}

\begin{sreoutput}
# Stage 1: Determining SSU start/end positions and best-matching models...
#
# output file name             description                                        
# ---------------------------  ----------------------------------------------
  myseqs2.tab                  locations/scores of hits defined by HMM(s)
  myseqs2.firmicutes.hitlist   list of sequences to align with firmicutes CM
  myseqs2.firmicutes.fa              1 sequence  to align with firmicutes CM
  myseqs2.bac-nonfirm.hitlist  list of sequences to align with bac-nonfirm CM
  myseqs2.bac-nonfirm.fa             4 sequences to align with bac-nonfirm CM
  myseqs2.archaea.hitlist      list of sequences to align with archaea CM
  myseqs2.archaea.fa                 5 sequences to align with archaea CM
  myseqs2.eukarya.hitlist      list of sequences to align with eukarya CM
  myseqs2.eukarya.fa                 5 sequences to align with eukarya CM
\end{sreoutput}

Earlier in this tutorial, we used the default models to classify and
align these sequences and found that 5 sequences were assigned to each of the
three domains. In this search, 1 of the 5 bacterial sequences has been
classified as a \emph{Firmicutes} sequence. The
\prog{myseqs2.firmicutes.hitlist} file lists the name of the
sequence:

\begin{sreoutput}
# target name                        start    stop     score
# --------------------------------  ------  ------  --------
  01106::Bacillus_subtilis::K00637       1    1552   2304.65
\end{sreoutput}

This is the \emph{Bacillus subtilis} sequence from the bacterial seed
that we purposefully omitted from the \emph{Firmicutes}
seed. \prog{ssu-align} has correctly identified it as a
\emph{Firmicutes} sequence. 
We omitted the sequence to make this task more challenging.
If we had included this sequence in the seed, its correct
identification would have been easier for the program.

In this section we've crudely split the bacterial seed alignment
into two smaller seed alignments and built a separate model from each,
without modifying either seed. A better, but more labor intensive,
strategy would be to manually examine and modify the seeds before constructing a
CM from them. Given the limited context of the sequences in the seed,
it may be clear that certain alignment regions should be refined and
that basepairs in the consensus secondary structure should be added or
removed. Also, more sequences could be added to make the model more
robust and more fully represent the \emph{Firmicutes} phyla. While six
sequences is probably too few, many more than 100 sequences in a seed
is typically not necessary.



