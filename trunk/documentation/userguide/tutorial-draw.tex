\subsection{Visualizing alignments with ssu-draw}

Because SSU rRNA is a long RNA (about 1500 nucleotides in archaea and
bacteria, and about 1800 in eukarya) alignments of even a few
sequences are difficult to view in a meaningful way. This is
especially true if the alignments contain thousands of sequences.  The
\prog{ssu-draw} program can be used to generate secondary structure
diagrams for displaying alignment statistics, such as location and
frequency of insertions and deletions, as well as individual aligned
sequences. \prog{ssu-draw} can only be used for alignments created
using the three default models; it cannot be used for alignments to
models created using \prog{ssu-build}.  To draw alignment statistic
diagrams of our \prog{myseqs} example dataset, do:

\user{ssu-draw myseqs}

Two new files will be created for each of the alignments, as reported
to the screen:

\begin{sreoutput}
# Drawing secondary structure diagrams...
#
# alignment file name  structure diagram file  # pages  tabular data file      
# -------------------  ----------------------  -------  -----------------------
  myseqs.archaea.stk   myseqs.archaea.pdf            9  myseqs.archaea.drawtab 
  myseqs.bacteria.stk  myseqs.bacteria.pdf           9  myseqs.bacteria.drawtab
  myseqs.eukarya.stk   myseqs.eukarya.pdf            9  myseqs.eukarya.drawtab 
\end{sreoutput}

The structure diagram files will be PDF format only if you have the
program \prog{ps2pdf} installed and in your PATH, otherwise they will
be postscript files (see the \prog{ssu-draw} manual page for more
information). 

Take a look at the nine page \prog{myseqs.archaea.pdf} or
\prog{myseqs.archaea.ps} file \footnote{For Mac OS/X users: the standard
  Mac Preview application should be able to view either type of
  file.}.  Page 2 of this file is included in this guide as
Figure~\ref{fig:myseqs-archaea-info}.

Each page shows the same secondary structure template, it is 1508
consensus positions for archaea and includes 471 basepairs as
indicated at the top of each page. Each of the nine pages displays a
different alignment statistic on each of these 1508 consensus
positions. Page 1 shows the alignment consensus sequence which is
defined by the most common nucleotide present in the alignment
\prog{myseq.archaea.stk} at each consensus position. Page 2 shows the
information content per consensus position in a heatmap-like color
scheme; blue positions are highly conserved, red positions are higly
variable. Page 3 shows the mutual information from basepaired
nucleotides in a similar heatmap style: paired positions that are red
exhibit many covariations across the sequences, blue positions exhibit
few covariations. Page 4 and 5 show the frequency and average length
of insertions, respectively, that occur after each consensus
position. An inserted nucleotide is one that does not align to a
consensus position.
Importantly, these nucleotides are not aligned by
\prog{ssu-align} but rather simply inserted between the two
appropriate consensus columns (see ~ref{sec:background-columns} for
more detail). Pages 6 shows the frequency of all
deletions (gaps) in the alignment. Page 7 shows the frequecy of
\emph{internal} deletions, these are deletions that occur after the first
nucleotide and before the final nucleotide in their corresponding
sequence. Page 8 shows the average posterior probability (confidence
estimate) per consensus position. Finally, page 9 shows the fraction
of sequences that span each position, i.e. that have a nongap
nucleotide occuring at or before the position and a nongap nucleotide
occuring at or after the position.

You may have noticed that while the majority of positions are drawn as
square cells, some of the positions are open circles. These are the
positions that are excluded by the mask we created earlier using the
command \prog{ssu-mask myseqs}. As it has done here, 
\prog{ssu-draw} will automatically use masks to show excluded
positions if \prog{ssu-mask} has already been executed
for the directory. To use the default masks discussed in the masking
section of this tutorial, use the \prog{-d} option (\prog{ssu-mask -d
  myseqs}), but note that this will overwrite the files we've just
created above unless you also use the \prog{--key-out} option
(e.g. \prog{ssu-mask --key-out default -d myseqs}) as shown previously in
the masking section.

By default, alignment summary diagrams are drawn as solid
color cells per position, but the consensus sequence can be overlaid
on the colored cells using the \prog{--cnt} option to \prog{ssu-draw}.

The \prog{.drawtab} suffixed files include tabular delimited data
corresponding to the statistics shown in the structure diagrams. In
these files, lines that start with \prog{\#} are comment lines
which are not tabular, but rather explain each tab-delimited field. The
comments also explain how the statistics, such as information content
and mutual information, are calculated.

\begin{figure}
  \begin{center}
\includegraphics[width=5.7in]{Figures/myseqs-archaea-info}
  \end{center}
\caption{\textbf{Information content per consensus position of the
          archaeal alignment created in the tutorial.} Coloring
  reflects level of conservation: highly conserved positions are blue,
  highly variable positions are red, as indicated in the
  legend. Solid square positions are included by the alignment mask
  created by \prog{ssu-mask}; open circle positions are excluded. See
  text for more details.}
\label{fig:myseqs-archaea-info}
\end{figure}

\subsubsection{Drawing individual sequences}

In addition to drawing alignment summary diagrams, \prog{ssu-draw} can
be used to create secondary structure diagrams of individual aligned
SSU sequences using the \prog{--indi-all} option:

\user{ssu-draw --indi-all myseqs}

As before, this command will create the nine-page alignment summary
diagram files, but it will also create \prog{indiall.pdf} suffixed
files which display each sequence in the alignment. 

Take a look at \prog{myseqs.archaea.indiall.pdf} or
\prog{myseqs.archaea.indiall.ps}. This file has 10 pages, two per
sequence. Pages 7 and 8, which display a Thermococcus sequence, are
included as Figures~\ref{fig:myseqs-archaea-indi-1} and
\ref{fig:myseqs-archaea-indi-2}.

The first page for each sequence is the aligned sequence with
nucleotides colored by basepair type, and background cells colored by
number of inserts after each nucleotide (blank (white) backgrounds
predominate, indicating 0 inserted nts after those
positions). Additionally, nucleotides that differ from the most
frequent nucleotide at each position are outlined, with thicker
outlines indicating the most frequent nucleotide at the position
occurs in more than 75\% of the aligned sequences. The legend section
explains the coloring and other formatting of each cell.

The second page for each sequence occurs immediately after
the first, and displays the posterior probability of each position
using a heatmap scheme: blue positions have high posterior
probability meaning the program is confident these positions are
correctly aligned, while red indicates low confidence and high
ambiguity. 

Be warned that these files can grow very large for large alignments,
especially if they are being created as postscript
files \footnote{When \prog{ssu-draw} creates PDF files it first
  creates postscript files, then converts them to PDF with
  \prog{ps2pdf}, and finally removes the postscript files.}. In
general, the files will be about 1 Mb per 20 sequences for PDFs and 1
Mb per 2 sequences for postscripts. By default, \prog{ssu-draw} will
exit with an error if the output postscript file you're requesting
would exceed 100 Mb in size, to override this warning and create the
file anyway, use the \prog{-f} option.

You might not want to draw all the sequences in the alignment,
especially since the output image files are large. To select a subset
of sequences to draw, use \prog{ssu-mask} with the \prog{--seq-k <listfile>} 
option to create a new alignment with only sequences listed in the
file \prog{<listfile>}
\footnote{It might be helpful to use \prog{ssu-mask --list myseqs} to
create lists of all the sequences in each alignment, and then remove
those you don't want.}. 
In a list file, each sequence name appears on
a separate line. An example list of 2 bacterial sequences from
the \prog{myseqs} dataset is in \\ 
\prog{tutorial/mytwobac.list}. To draw just
these two sequences, you would execute the following two commands:

\user{ssu-mask -a --seq-k mytwobac.list myseqs/myseqs.bacteria.stk} 

\user{ssu-draw -a --indi myseqs.bacteria.seqk.stk}

Notice that the \prog{-a} option was also used and instead of using
\prog{myseqs} as the command-line argument we specified the bacterial
alignment. The \prog{-a} option tells \prog{ssu-draw} that the
command-line argument is an alignment file, not a
\prog{ssu-align}-created directory.

There are several other options that can be supplied to
\prog{ssu-draw}. See the \prog{ssu-draw} manual page at the end of
this guide for more information.

\begin{figure}
  \begin{center}
\includegraphics[width=5.7in]{Figures/myseqs-archaea-indi-1}
  \end{center}
\caption{\textbf{Sequence and predicted secondary structure of
    a \emph{Thermococcus celer} SSU sequence as it aligns to the default
    archaeal model.} Colors and other annotations of the nucleotides
  are listed in the legend and discussed in the text.}
\label{fig:myseqs-archaea-indi-1}
\end{figure}

\begin{figure}
  \begin{center}
\includegraphics[width=5.7in]{Figures/myseqs-archaea-indi-2}
  \end{center}
\caption{\textbf{Sequence and posterior probabilities for a
    \emph{Thermococcus celer} SSU sequence as it aligns to the default
    archaeal model.} Colors and other annotations of the nucleotides
  are listed in the legend and discussed in the text.}
\label{fig:myseqs-archaea-indi-2}
\end{figure}
