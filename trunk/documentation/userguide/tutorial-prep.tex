\subsection{Faster alignment of large datasets through parallelization
  with ssu-prep}
\label{sec:tutorial-prep}
\prog{ssu-align} runs on a single processor and does not support MPI
or multi-threading. However, if you have access to a compute
cluster or multi-core computers, simplistic parallelization is possible
with the \prog{ssu-prep} program.

\prog{ssu-prep} will split up a large input sequence file into
\emph{n} smaller files and create a shell script that will execute
\emph{n} \prog{ssu-align} jobs in parallel, each processing one of the
small sequence files. The results of all jobs will automatically be
merged together by the final job, ultimately yielding the same results
as if a single 
\prog{ssu-align}
job was run for the original large input sequence file. 
Parallelizing \prog{ssu-align} in this way can drastically reduce the
actual time required for aligning large datasets. A job that would
have required 100 hours on 1 processor can be done in a little more
than 1 hour on 100 processors. See table~\ref{tbl:ttimes} in
section~\ref{sec:stats} for example timing statistics. 

In this section we'll walk through an example of how to do this for a
small dataset.  The sequence file \prog{tutorial/seed-30.fa} includes
30 randomly chosen sequences from the complete set of seed sequences
from the three default models of SSU-ALIGN.

\prog{ssu-prep} has three different usage modes as explained by the
program if it is run without any command-line arguments:

\user{ssu-prep}

\begin{sreoutput}
Incorrect number of command line arguments.
Usage: ssu-prep    [-options] <seqfile> <output dir> <num jobs> <prefix/suffix file>
Usage: ssu-prep -x [-options] <seqfile> <output dir> <num jobs>
Usage: ssu-prep -y [-options] <seqfile> <output dir> <num jobs>

ssu-prep splits up <seqfile> into <num jobs> smaller files and creates a shell
script that will execute <num jobs> ssu-align jobs in parallel, each processing
one of the small sequence files. The results of all jobs will automatically be
merged together by the final job, giving the same results as if a single
ssu-align job was run for the complete <seqfile>.

The 3 different usages control how the prefix and suffix are defined for the jobs
in the output shell script, allowing, for example, the user to wrap the ssu-align
commands in a cluster submission command (such as Sun Grid Engine's "qsub"):

Default: (neither -x nor -y enabled) prefix and suffix for ssu-align jobs in
         output shell script are defined in <prefix/suffix file>. First line is
         the prefix, second line is the suffix.
With -x: do not specify <prefix/suffix file>; output shell script will run all
         <num jobs> jobs in parallel on one machine with <num jobs> cores/cpus.
With -y: do not specify <prefix/suffix file>; user will manually add the desired
         prefix/suffix to ssu-align commands after ssu-prep finishes.

To see more help on available options, do ssu-prep -h
\end{sreoutput}

By default, if neither \prog{-x} nor \prog{-y} options are used, the program
reads a \prog{<prefix/suffix file>}, a simple two line file that
a prefix string and a suffix string to prepend and append respectively
to the \prog{ssu-align} commands it generates. 

The strings in the \prog{<prefix/suffix file>} will likely be specific
to your parallel computing environment. At Janelia, we use Sun Grid
Engine's \prog{qsub} program for submitting jobs to a large compute
cluster. The relevant prefix/suffix file for our specific
computing environment is included in \\
\prog{tutorial/janelia-cluster-presuf.txt}. Take a look at this file:

\begin{sreoutput}
qsub -N ssu-align -o /dev/null -b y -j y -cwd -V "
"
\end{sreoutput}

The first line is the prefix string, containing the name and
appropriate command line arguments of the \prog{qsub} program which
submits jobs to the Grid Engine queuing system at Janelia. The second
line is the suffix, and contains only a double quote, which
complements the double quote at the end of the prefix line as we'll
see below.

Now, let's run \prog{ssu-prep} as if we were going to create parallel
\prog{ssu-align} jobs for the Janelia cluster.  Move into the
\prog{tutorial/} directory and execute the command:

\user{ssu-prep seed-30.fa my30 5 janelia-cluster-presuf.txt}

About forty lines of text are output to the screen. We'll step through
and discuss this output: 

\begin{sreoutput}
# Validating input sequence file ... done.
#
# Preparing 5 ssu-align jobs ...
# Partitioning seqs with goal of equalizing total number of nucleotides per job ...
#
# output file name   description                                       
# -----------------  --------------------------------------------------
  my30/seed-30.fa.1  partition 1 FASTA sequence file (6 seqs; 10463 nt)
  my30/seed-30.fa.2  partition 2 FASTA sequence file (6 seqs; 10240 nt)
  my30/seed-30.fa.3  partition 3 FASTA sequence file (6 seqs;  9922 nt)
  my30/seed-30.fa.4  partition 4 FASTA sequence file (6 seqs; 10358 nt)
  my30/seed-30.fa.5  partition 5 FASTA sequence file (6 seqs;  9397 nt)
  my30.ssu-align.sh  shell script that will execute 5 ssu-align jobs
#
\end{sreoutput}

First, \prog{ssu-prep} reports that it has validated the formatting of
the sequence file, partitioned it into 5 new files and placed each file
into the newly created \prog{my30/} subdirectory. Each of these files has
6 of the original 30 sequences in it. Additionally,
\prog{my30.ssu-align.sh}, a shell script that will execute 5 jobs, one
per sequence file, was created in the current working directory. After
this, you'll see:

\begin{sreoutput}
################################################################################
# To execute all 5 ssu-align jobs, run the shell script with:
#	sh my30.ssu-align.sh
################################################################################
\end{sreoutput}

These are instructions for how to execute the shell script. Take a
look at the shell script \\ \prog{my30.ssu-align.sh}:

\begin{sreoutput}
#!/bin/bash
# Bash shell script created by ssu-prep for running 5 ssu-align jobs.
# Each job will process a separate partition of the sequence file:
# 'seed-30.fa'.
#
# The final job is special, after computing its alignments it will wait for all
# other jobs to finish and then merge the output of all jobs together.
# The merged output files will be in the directory: '/my30/'
#
# The for loop below will execute/submit the first 4 of 5 jobs.
# The final ssu-align job is executed separately because it does the merging.
#
for i in {1..4}
do
	echo "# Executing: qsub -N ssu-align -o /dev/null -b y -j y -cwd -V " ssu-align my30/seed-30.fa.$i my30/my30.$i ""
	qsub -N ssu-align -o /dev/null -b y -j y -cwd -V " ssu-align my30/seed-30.fa.$i my30/my30.$i "
done
echo "# Executing: qsub -N ssu-align -o /dev/null -b y -j y -cwd -V " ssu-align --merge 5 my30/seed-30.fa.5 my30/my30.5 ""
qsub -N ssu-align -o /dev/null -b y -j y -cwd -V " ssu-align --merge 5 my30/seed-30.fa.5 my30/my30.5 "
\end{sreoutput}

This is a bash shell script file\footnote{The \prog{--no-bash} option
  can be used to make a non-bash-specific script, see the
  \prog{ssu-prep} manual page for more information.}.
The \prog{\#}-prefixed lines are
explanatory comments. The remainder of the file consists of a for loop
that will submit the first four \prog{ssu-align} jobs to the cluster
using \prog{qsub}. The lines beginning with \prog{qsub} are the actual
job submission commands. The lines beginning with \prog{echo} cause updates to
be printed to STDOUT before each job is submitted. 
Note that the \prog{qsub} command line is composed
of the prefix string from the \prog{janelia-cluster-presuf.txt} file,
followed by a \prog{ssu-align} command, followed by the suffix
string. The end result is that the \prog{ssu-align} command is
contained within the quotes from the prefix/suffix strings.

The comments explain that the final job is special. It will merge the
results of all jobs once they are finished. Consequently, it requires
special command-line options and so is executed outside the for loop,
as the final line of the script.

%%%%%%%%%%%%%%%%%%%%%%%%%%%%%%%%%%%%%%%%%%%%%%%%%%%%%%%%%%%%%%%%%%%%%
\begin{comment}
The final important part of the \prog{ssu-prep} output explains what
to do if any of the jobs fail: 

\begin{sreoutput}
# If one or more jobs fail: rerun the failed jobs, wait for them to finish,
# and then perform manual merge from this directory by executing:
#	ssu-merge my30
\end{sreoutput}

This should happen only rarely, but if any jobs fail, this aspect of
the program allows 
\end{comment}
%%%%%%%%%%%%%%%%%%%%%%%%%%%%%%%%%%%%%%%%%%%%%%%%%%%%%%%%%%%%%%%%%%%%%

Because your specific compute system is likely different from
Janelia's, the \prog{my30.ssu-align.sh} script will probably not
work. To make \prog{ssu-prep} generate shell scripts you can run on
your system, create a file like \prog{janelia-cluster-presuf.txt} but
with prefix and suffix strings specific to your
system. 

This simple prefix/suffix string method may not work for your
compute system. For example, if your cluster requires using \prog{ssh}
to remote login to different nodes, a single fixed prefix line will
probably not be sufficient. If this is the case, run \prog{ssu-prep}
with the \prog{-y} option. In this mode no prefix/suffix file
will be read, and the output shell script will contain only the raw
\prog{ssu-align} commands. For example, do:

\user{ssu-prep -f -y seed-30.fa my30 5}

(We had to also specify \prog{-f} so the program would overwrite our
existing \prog{my30} directory.) Much of the output is the same as the
prior example, but the \prog{WARNING} section is new:

\begin{sreoutput}
################################################################################
# WARNING: -y was set on the command line.
# This means that 'my30.ssu-align.sh' will simply run the 5 ssu-align jobs in
# succession, one after another, not in parallel. If you want to run the jobs in
# parallel you'll either have to manually edit that file or rerun ssu-prep using
# the options listed above to specify prefix and/or suffix strings for the
# ssu-align commands, so they are, for example, submitted to run on a cluster
# using a queing system manager like SGE. Or you can run <n> jobs in parallel on
# a single <n>-core machine by rerunning ssu-prep using '-x'.
# Do 'ssu-prep -h' or see the User Guide for more information.
################################################################################
\end{sreoutput}

As explained, you'll have to modify the newly created
\prog{my30.ssu-align.sh} to make the jobs run in parallel in this
case, either manually or using your own script. The file will still
contain a for loop submitting all but the final job. If you'd rather
all jobs were placed on their own line, use the \prog{--no-bash}
option to \prog{ssu-prep}.

For now, let's run the unmodified \prog{my30.ssu-align.sh} script to
demonstrate how the jobs are merged together automatically. As noted
in the warning above, this will simply run the 5 jobs in succession
instead of in parallel, but for our purposes here this is okay. To run
the script, execute the command: 

\user{sh my30.ssu-align.sh \footnote{If this command fails, you're
    probably not using the BASH shell. Rerun the \prog{ssu-prep}
    command above with the \prog{--no-bash} option and try again.}}

The script executes \prog{ssu-align} five times in succession, and
these jobs will being reporting what they're doing to the screen. 
Once all five jobs are run, the \prog{ssu-merge} program will
automatically be called to merge their output. You'll see the
following output at that point:

\begin{sreoutput}
# Merging files from 5 ssu-align runs...
#
#                                  # files     # seqs
# merged file name       CM name    merged     merged
# ---------------------  --------  -------  ---------
  my30.tab               -               5          -
  my30.scores            -               5          -
  my30.ssu-align.sum     -               5          -
  my30.ssu-align.log     -               5          -
#
  my30.archaea.fa        archaea         2          5
  my30.archaea.hitlist   archaea         2          5
  my30.archaea.cmalign   archaea         2          5
  my30.archaea.ifile     archaea         2          5
  my30.archaea.stk       archaea         2          5
#
  my30.bacteria.fa       bacteria        5          8
  my30.bacteria.hitlist  bacteria        5          8
  my30.bacteria.cmalign  bacteria        5          8
  my30.bacteria.ifile    bacteria        5          8
  my30.bacteria.stk      bacteria        5          8
#
  my30.eukarya.fa        eukarya         5         17
  my30.eukarya.hitlist   eukarya         5         17
  my30.eukarya.cmalign   eukarya         5         17
  my30.eukarya.ifile     eukarya         5         17
  my30.eukarya.stk       eukarya         5         17
\end{sreoutput}

Two of the five small partition files included archaeal sequences, and
all five included bacterial and eukaryotic  sequences. The total number
of archaeal, bacterial and eukaryotic sequences was 5, 8 and 17,
respectively. The final merged alignments are in the files
\prog{my30.archaea.stk}, \prog{my30.bacteria.stk}, and
\prog{my30.eukarya.stk}.

subsubsection{Using ssu-prep to parallelize ssu-align
  on multi-core machines}

The third and final \prog{ssu-prep} usage mode is for parallelizing
jobs to run on a single multi-core machine. This mode can be thought
of as a simple substitute for multi-threading, and is enabled with the
\prog{-x} option to \prog{ssu-prep}. As with \prog{-y}, using
\prog{-x} obviates the need for a prefix/suffix file. As an example,
imagine you are using a quad-core machine. In this case, execute: 

\user{ssu-prep -f -x seed-30.fa my30 4}

(We had to also specify \prog{-f} so the program would overwrite our
existing \prog{my30} directory.) Much of the output is the same as
before. However, the \prog{my30.ssu-align.sh} file will be different;
it is included below:

\begin{sreoutput}
#!/bin/bash
# Bash shell script created by ssu-prep for running 4 ssu-align jobs.
# Each job will process a separate partition of the sequence file:
# 'seed-30.fa'.
#
# This script will execute all 4 jobs at once, in parallel. It is only
# meant to be executed on a system with 4 cpus/cores. The first 3 jobs
# will run in the background and output to /dev/null. The final job will
# output to STDOUT, allowing you to follow its progress.
#
# The final job is special, after computing its alignments it will wait for all
# other jobs to finish and then merge the output of all jobs together.
# The merged output files will be in the directory: '/my30/'
#
# The for loop below will execute/submit the first 3 of 4 jobs.
# The final ssu-align job is executed separately because it does the merging.
#
for i in {1..3}
do
	echo "# Executing: ssu-align my30/seed-30.fa.$i my30/my30.$i > /dev/null &"
	ssu-align my30/seed-30.fa.$i my30/my30.$i > /dev/null &
done
echo "# Executing: ssu-align --merge 4 my30/seed-30.fa.4 my30/my30.4"
ssu-align --merge 4 my30/seed-30.fa.4 my30/my30.4
\end{sreoutput}

Note that the \prog{ssu-align} commands in the for loop include
\prog{\&} at the end, which cause them to be run simultaneously.









