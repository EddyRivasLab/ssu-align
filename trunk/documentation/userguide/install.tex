% EPN, Thu Jun  3 08:04:50 2010
% EPN, Mon Feb 22 09:31:56 2016 [updated for 0.1.1 release]
% NOTE: this install section was created by taking the first half of
% the HMMER3 installation section and the second half of the Infernal
% installation sections, glueing them together and making the
% necessary changes: substituting file names, adding text and removing
% parts that either don't apply, or don't work in SSU-ALIGN.

\section{Installation}
\label{sec:install}

\subsection{Quick installation instructions}
Download the source tarball (\prog{ssu-align-0.1.1.tar.gz}) from 
\url{http://eddylab.org/software/ssu-align}, or directly from 
\url{eddylab.org/software/ssu-align/ssu-align-0.1.1.tar.gz}; \\
untar; and change into the newly created directory \prog{ssu-align-0.1.1}:

\user{wget eddylab.org/software/ssu-align/ssu-align-0.1.1.tar.gz}; \\
\user{tar xfz ssu-align-0.1.1.tar.gz}\\
\user{cd ssu-align-0.1.1}

Configure for your system, and build the programs:

\user{./configure}\\
\user{make}

\begin{comment}
At this point, the programs are in the \prog{src} subdirectory, the
\prog{infernal/src} subdirectory and the
\prog{infernal/easel/miniapps} subdirectory. The
user's guide (this document) is in the \\ 
\prog{documentation/userguide}
subdirectory. The man pages are in the \prog{documentation/manpages}
subdirectory. You can manually move or copy all of these to
appropriate locations if you want. You will want the programs to be in
your \$PATH. 
\end{comment}

Install the man pages and programs in system-wide
directories:

\user{make install}

This will install three types of files: programs, man pages, and data
files.  By default, programs are installed in
\prog{/usr/local/bin}. Installed programs include the six main
SSU-ALIGN scripts (\prog{ssu-align}, \prog{ssu-build},
\prog{ssu-draw}, \prog{ssu-mask}, \prog{ssu-merge} and
\prog{ssu-prep}), the INFERNAL programs, each of which is prefixed
with 'ssu-cm', e.g. \prog{ssu-cmsearch}, and EASEL programs called
\emph{miniapps}, these are all named with a 'ssu-esl' prefix,
e.g. \prog{ssu-esl-seqstat}. The INFERNAL and EASEL programs are
called internally by the SSU-ALIGN scripts. Normal users will probably
never need to run them.  By default, man pages for all programs are
installed in \prog{/usr/local/man/man1}. The file \prog{ssu.1} in that
directory is a special man page that summarizes the entire package.
Finally, data files and a PERL module (\prog{ssu.pm}) required by the
programs are placed in \prog{/usr/local/share/ssu-align-0.1.1}, by
default. Data files include the default CM files, the seed alignments,
mask files, and postscript secondary structure template files.

You might need special privileges to write to \prog{/usr/local}. (You may need to execute
\prog{sudo make install}). You can change \prog{/usr/local} to any
directory you want using the \prog{./configure --prefix} option, as in
\prog{./configure --prefix /the/directory/you/want}.

After installing, you'll see the following:

\begin{sreoutput}
=====================================
SSU-ALIGN has been successfully built
=====================================

The final step is to update your environment variables:

If you are using the bash shell, add the following three:
lines to the '.bashrc' file in your home directory:

export PATH="\$PATH:/usr/local/bin"
export MANPATH="\$MANPATH:/usr/local/share/man"
export SSUALIGNDIR="/usr/local/share/ssu-align-0.1.1"

And then source that file to update your current
environment with the command:

source ~/.bashrc

If you are using the C shell, add the following three:
lines to the '.cshrc' file in your home directory:

setenv PATH "\$PATH:/usr/local/bin"
setenv MANPATH "\$MANPATH:/usr/local/share/man"
setenv SSUALIGNDIR "/usr/local/share/ssu-align-0.1.1"

And then source that file to update your current
environment with the command:

source ~/.cshrc

(To determine which shell you use, type: 'echo \$SHELL')
\end{sreoutput}

Follow these instructions to update your environment for the current
and every subsequent shell session. This step is crucial because it
ensures all the required executables are in your PATH. SSU-ALIGN
internally calls INFERNAL and EASEL programs that were just installed,
all of which must be in your PATH.

For instance, if you use the bash shell you'd 
open your \prog{.bashrc} file in your home directory (e.g.
\prog{Users/nawrockie/.bashrc}, or just \prog{~/.bashrc}) and add the
three lines: 

\prog{export PATH="\$PATH:/usr/local/bin"}

\prog{export MANPATH="\$MANPATH:/usr/local/share/man"}

\prog{export SSUALIGNDIR="/usr/local/share/ssu-align-0.1.1"}

Note that these will only work for the current configuration (that is,
they'll be different if you change the installation locations with
\prog{--prefix} or something like it). Then, source this file to
update the current session:

\user{source $\sim$/.bashrc}

After doing this, you can test if your environment is setup correctly
by checking the following two commands: 

\user{which ssu-align} 

should return the path to the \prog{ssu-align} script you
just installed (\prog{/usr/local/bin/ssu-align} if you configured
with default settings), and

\user{echo \$SSUALIGNDIR}

should return the path to the your data directory
(\prog{/usr/local/share/ssu-align-0.1.1} if you configured with default
settings).

If you want to, at this point you can run the automated
testsuite. (You can't run the testsuite before installation because it
requires all the INFERNAL and EASEL programs be in your path.)
This step is optional. It takes about 10 minutes, and all these
tests should pass: 

\user{make check}

That's it. You can keep reading if you want to know more about
customizing a SSU-ALIGN installation, or you can skip ahead to
section~\ref{sec:tutorial}, the tutorial.

\subsection{More detailed installation notes}

The six main programs in the SSU-ALIGN package:
\prog{ssu-align}, \prog{ssu-build}, \prog{ssu-draw}, \prog{ssu-mask},\\
\prog{ssu-merge} and \prog{ssu-prep} are PERL scripts. The
package depends on, and includes, INFERNAL 
which itself includes HMMER and the EASEL sequence analysis
library; all of which are distributed as ANSI C source code. 

SSU-ALIGN is designed to be built and used on UNIX platforms. It
is developed mainly on Intel GNU/Linux systems and Mac OS, and
intermittently tested on a variety of other UNIX platforms. It is not
currently tested on Microsoft Windows, but it should work there; it
should be possible to build it on any platform with an ANSI C
compiler. The software itself is vanilla POSIX-compliant ANSI C and
PERL. You may need to work around the configuration scripts and
Makefiles to get it built on a non-UNIX platform.

The GNU configure script that comes with SSU-ALIGN has a
number of options. You can see them all by doing:

\user{./configure --help}

Be warned that some of these options (such as \prog{--enable-mpi}) will
\emph{not} affect SSU-ALIGN performance. These options affect only how
INFERNAL or HMMER is built underneath SSU-ALIGN,
but have no impact on the toplevel SSU-ALIGN programs. 

All customizations can and should be done at the \prog{./configure}
command line, unless you're a guru delving into the details of the
source code.

\subsubsection{Setting installation targets}

The most important options are those that let you set the installation
directories for \prog{make install} to be appropriate to your system.
What you need to know is that SSU-ALIGN installs three types
of files: programs, man pages, and data files. It installs the programs in
\prog{--bindir} (which defaults to \prog{/usr/local/bin}), the man pages in the
\prog{man1} subdirectory of \prog{--mandir} (default
\prog{/usr/local/man}) and the data files in a newly created
subdirectory \prog{ssu-align-0.1.1} of \prog{--datadir} (which
defaults to \prog{/usr/local/share}. Thus, say you want
\prog{make install} to install programs in \prog{/usr/bioprogs/bin}, man pages in
\prog{/usr/share/man/man1} and data files in \prog{/usr/share/ssu-align-0.1.1};
you would configure with:

\noindent \prog{> ./configure --mandir=/usr/share/man --bindir=/usr/bioprogs/bin --datatdir=/usr/share}

That's really all you need to know, since SSU-ALIGN installs
so few files. But just so you know; GNU configure is very flexible,
and has shortcuts that accomodates several standard conventions for
where programs get installed. One common strategy is to install all
files under one directory, like the default \prog{/usr/local}. To
change this prefix to something else, say \prog{/usr/mylocal}
(so that programs go in \prog{/usr/mylocal/bin}, man pages in
\prog{/usr/mylocal/man/man1}, and data files in
\prog{/usr/mylocal/share/ssu-align-0.1.1}), you can use the
\prog{--prefix} option:

\user{./configure --prefix=/usr/mylocal}

\begin{comment}
% EPREFIX doesn't work with ssu-align's configure; not sure why
Another common strategy (especially in multiplatform environments) is
to put programs in an architecture-specific directory like
\prog{/usr/share/Linux/bin} while keeping man pages in a shared,
architecture-independent directory like \prog{/usr/share/man/man1}.
GNU configure uses \prog{--exec-prefix} to set the path to
architecture dependent files; normally it defaults to being the same
as \prog{--prefix}. You could change this, for example, by:

\user{./configure --prefix=/usr/share --exec-prefix=/usr/share/Linux}\\
\end{comment}

In summary, a complete list of the \prog{./configure} installation
options that affect SSU-ALIGN:

\begin{tabular}{lll}
Option                       & Meaning                         & Default\\ \hline
\prog{--prefix=PREFIX}       & all files                       & \prog{/usr/local} \\
\prog{--bindir=DIR}          & programs                        & PREFIX/bin/\\
\prog{--mandir=DIR}          & man pages                       & PREFIX/man/\\
\prog{--datadir=DIR}         & data files                      & PREFIX/share/\\
\end{tabular}


\subsubsection{Setting compiler and compiler flags}

By default, \prog{configure} searches first for the GNU C compiler
\prog{gcc}, and if that is not found, for a compiler called \prog{cc}. 
This can be overridden by specifying your compiler with the \prog{CC}
environment variable.

By default, the compiler's optimization flags are set to
\prog{-g -O3} for \prog{gcc}, or \prog{-g} for other compilers.
This can be overridden by specifying optimization flags with the
\prog{CFLAGS} environment variable. 

For example, to use an Intel C compiler in
\prog{/usr/intel/ia32/bin/icc} with 
optimization flags \prog{-O3 -ipo}, you would do:

\user{env CC=/usr/intel/ia32/bin/icc CFLAGS="-O3 -ipo" ./configure}

which is the one-line shorthand for:

\user{setenv CC     /usr/intel/ia32/bin/icc}\\
\user{setenv CFLAGS "-O3 -ipo"}\\
\user{./configure}

If you are using a non-GNU compiler, you will almost certainly want to
set \prog{CFLAGS} to some sensible optimization flags for your
platform and compiler. The \prog{-g} default generated unoptimized
code. At a minimum, turn on your compiler's default optimizations with
\prog{CFLAGS=-O}.

% This is copied and modified from the HMMER3 installation section of
% its user's guide
\subsection{Makefile targets}

\begin{sreitems}{\emprog{distclean}}
\item[\emprog{all}]
  Builds everything. Same as just saying \prog{make}.

\item[\emprog{check}]
  Runs the automated test for SSU-ALIGN.

\item[\emprog{clean}]
  Removes all files generated by compilation (by
  \prog{make}). Configuration (files generated by
  \prog{./configure}) is preserved.

\item[\emprog{distclean}]
  Removes all files generated by configuration (by \prog{./configure})
  and by compilation (by \prog{make}). 

  Note that if you want to make a new configuration you
  should do a \prog{make distclean} (rather than a \prog{make
    clean}), to be sure old configuration files aren't used
  accidentally.
\end{sreitems}

