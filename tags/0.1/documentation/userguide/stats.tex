\section{Running times and output file sizes for example datasets}
\label{sec:stats}

This section contains timing and output file size statistics for
aligning, masking and drawing SSU sequences with the SSU-ALIGN
package to give you an idea of how much time and disk space you'll
need for processing various sized datasets.  The statistics are
displayed in tables~\ref{tbl:sizes}, \ref{tbl:ttimes}, and
\ref{tbl:ptimes} on the next two pages. The timing statistics reported
are for single execution threads running on 3.0 Intel Xeon
processors. All programs were run with default parameters (no
options), with the exception that some of the \emph{synthetic-v4} datasets
were processed with a truncated set of models as explained further
below.

%
%%%%%%%%%%%%%%%%%%%%%%%%%%%%%%%%%%%%%%%%
% Notes on how these tables were created are in
% /groups/eddy/home/nawrockie/notebook/10_0322_ssualign_auto_merge/
% This work, as logged in the 00LOG in that dir, was done on May
% 24-27, 2010.
%%%%%%%%%%%%%%%%%%%%%%%%%%%%%%%%%%%%
\begin{table}[hb]
\begin{center}
  \scriptsize
  \begin{tabular}{lllrr|rrrr|rr} 
                &          &          &         &        &       &       &       &           &           & masked \\
                &          &          &         &average &total  & \# of & \# of & \# columns& alignment & alignment \\
                &          &          & \# of   &sequence&\# of  &insert &consensus&included & file      & file \\
dataset         & models   & domain   &sequences& length &columns&columns&columns&by mask    & size (Mb) & size (Mb) \\ \hline
& & & & & & & & & & \\ 
Ggenes-core     & default  & archaea  &     186 & 1407.5 &  1757 &   249 &  1508 &  1380 &      0.7 &      0.5 \\
                & default  & bacteria &    4752 & 1438.1 &  5319 &  3737 &  1582 &  1362 &     50.7 &     13.1 \\
& & & & & & & & & & \\ 
Ggenes-full     & default  & archaea  &    8407 & 1365.0 &  3151 &  1643 &  1508 &  1382 &     53.3 &     23.5 \\
                & default  & bacteria &  499787 & 1400.3 & 11900 & 10318 &  1582 &  1404 & 11910.9 &   1419.4 \\
& & & & & & & & & & \\ 
Silva-Ref       & default  & archaea  &   19260 & 1169.0 & 12107 & 10599 &  1508 &  1383 &    469.2 &     56.1 \\
                & default  & bacteria &  391158 & 1413.0 & 24540 & 22958 &  1582 &  1396 & 19254.6 &   1148.6 \\
                & default  & eukarya  &   50350 & 1727.3 & 20950 & 19069 &  1881 &  1418 &   2117.0 &    150.1 \\
& & & & & & & & & & \\ 
Silva-Parc      & default  & archaea  &   56113 &  862.1 & 14299 & 12791 &  1508 &  1383 &   1613.0 &    163.0 \\
                & default  & bacteria & 1062938 & 1016.4 & 27918 & 26336 &  1582 &  1391 & 59498.1 &   3105.0 \\
                & default  & eukarya  &  123454 & 1092.6 & 24247 & 22366 &  1881 &  1424 &   6003.9 &    368.6 \\
& & & & & & & & & & \\ 
RDP             & default  & archaea  &   55249 &  830.9 & 12724 & 11216 &  1508 &  1382 &   1411.7 &    158.4 \\
                & default  & bacteria & 1182624 &  975.1 & 29736 & 28154 &  1582 &  1397 & 70455.3 &   3426.5 \\ %\hline
%& & & & & & & & & & \\ \hline
%& & & & & & & & & & \\ %\hline
%synthetic-full  & default  & archaea  &      34 & 1479.2 &  1596 &    88 &  1508 &  1400 &      0.113 &      0.099 \\
%synthetic-full  & default  & bacteria &      33 & 1530.0 &  1660 &    78 &  1582 &  1480 &      0.114 &      0.102 \\
%synthetic-full  & default  & eukarya  &      33 & 1805.7 &  2343 &   462 &  1881 &  1550 &      0.161 &      0.107 \\
%& & & & & & & & & & \\ 
%%synthetic-full  & default  & archaea  &     334 & 1479.7 &  2040 &   532 &  1508 &  1424 &      1.381 &      0.968 \\
%%synthetic-full  & default  & bacteria &     333 & 1526.9 &  1872 &   290 &  1582 &  1476 &      1.265 &      1.001 \\
%%synthetic-full  & default  & eukarya  &     333 & 1814.4 &  3312 &  1431 &  1881 &  1551 &      2.226 &      1.050 \\
%%& & & & & & & & & & \\ 
%synthetic-full  & default  & archaea  &    3334 & 1480.2 &  3781 &  2273 &  1508 &  1423 &     25.366 &      9.638 \\
%synthetic-full  & default  & bacteria &    3333 & 1526.8 &  2844 &  1262 &  1582 &  1466 &     19.117 &      9.929 \\
%synthetic-full  & default  & eukarya  &    3333 & 1814.0 &  4827 &  2946 &  1881 &  1554 &     32.333 &     10.509 \\
%& & & & & & & & & & \\ 
%%synthetic-full  & default  & archaea  &   33334 & 1480.3 &  6312 &  4804 &  1508 &  1424 &    422.354 &     96.472 \\
%%synthetic-full  & default  & bacteria &   33333 & 1526.5 &  5217 &  3635 &  1582 &  1465 &    349.407 &     99.269 \\
%%synthetic-full  & default  & eukarya  &   33333 & 1813.9 &  7888 &  6007 &  1881 &  1552 &    527.411 &    105.002 \\
%%& & & & & & & & & & \\ 
%synthetic-full  & default  & archaea  &  333334 & 1480.4 &  8800 &  7292 &  1508 &  1424 &   5882.696 &    965.338 \\
%synthetic-full  & default  & bacteria &  333333 & 1526.6 &  7798 &  6216 &  1582 &  1465 &   5215.344 &    993.335 \\
%synthetic-full  & default  & eukarya  &  333333 & 1814.4 & 11342 &  9461 &  1881 &  1551 &   7577.349 &   1050.002 \\ %\hline
%& & & & & & & & & & \\ %\hline
%synthetic-v4    & default  & archaea  &      50 &  238.7 &  1519 &    11 &  1508 &   241 &      0.157 &      0.027 \\
%synthetic-v4    & default  & bacteria &      50 &  239.7 &  1591 &     9 &  1582 &   237 &      0.164 &      0.026 \\
%& & & & & & & & & & \\ 
%%synthetic-v4    & default  & archaea  &     500 &  238.9 &  1603 &    95 &  1508 &   241 &      1.626 &      0.262 \\
%%synthetic-v4    & default  & bacteria &     500 &  239.4 &  1614 &    32 &  1582 &   238 &      1.637 &      0.259 \\
%%& & & & & & & & & & \\ 
%synthetic-v4    & default  & archaea  &    5000 &  238.9 &  1927 &   419 &  1508 &   243 &     19.484 &      2.641 \\
%synthetic-v4    & default  & bacteria &    5000 &  239.5 &  1823 &   241 &  1582 &   239 &     18.444 &      2.601 \\
%& & & & & & & & & & \\ 
%%synthetic-v4    & default  & archaea  &   50000 &  238.9 &  2301 &   793 &  1508 &   246 &    232.305 &     26.801 \\
%%synthetic-v4    & default  & bacteria &   50000 &  239.5 &  2168 &   586 &  1582 &   240 &    219.004 &     26.201 \\
%%& & & & & & & & & & \\ 
%synthetic-v4    & default  & archaea  &  500000 &  238.9 &  2667 &  1159 &  1508 &   254 &   2690.005 &    277.001 \\
%synthetic-v4    & default  & bacteria &  500000 &  239.5 &  2563 &   981 &  1582 &   244 &   2586.005 &    267.001 \\ %\hline
%& & & & & & & & & & \\ %\hline
%synthetic-v4    & v4       & archaea  &      50 &  239.3 &   254 &    13 &   241 &   241 &      0.028 &      0.027 \\
%synthetic-v4    & v4       & bacteria   &      50 &  239.7 &   250 &     9 &   241 &   237 &      0.027 &      0.026 \\
%& & & & & & & & & & \\ 
%%synthetic-v4    & v4       & archaea  &     500 &  239.6 &   346 &   105 &   241 &   241 &      0.367 &      0.262 \\
%%synthetic-v4    & v4       & bacteria   &     500 &  239.7 &   278 &    37 &   241 &   238 &      0.299 &      0.259 \\
%%& & & & & & & & & & \\ 
%synthetic-v4    & v4       & archaea  &    5000 &  239.6 &   686 &   445 &   241 &   241 &      7.071 &      2.621 \\
%synthetic-v4    & v4       & bacteria   &    5000 &  239.7 &   490 &   249 &   241 &   238 &      5.111 &      2.591 \\
%%& & & & & & & & & & \\ 
%%synthetic-v4    & v4       & archaea  &   50000 &  239.6 &  1075 &   834 &   241 &   241 &    109.702 &     26.301 \\
%%synthetic-v4    & v4       & bacteria   &   50000 &  239.7 &   865 &   624 &   241 &   238 &     88.702 &     26.001 \\
%& & & & & & & & & & \\ 
%synthetic-v4    & v4       & archaea   &  500000 &  239.6 &  1469 &  1228 &   241 &   241 &   1492.003 &    264.001 \\
%synthetic-v4    & v4       & bacteria    &  500000 &  239.7 &  1283 &  1042 &   241 &   238 &   1306.003 &    261.001 \\
\end{tabular}
\end{center}
\caption{\textbf{Statistics and resulting output file sizes for
    various SSU datasets.} For each dataset, the number
  of and average length of sequences for each domain, as classified by
  \prog{ssu-align}, is reported. For each dataset and domain an
  alignment is created. The four columns in the middle block report
  the total number of columns, the
  number of insert and consensus columns (consensus column counts are
  invariant per domain across all datasets), and the number of
  consensus columns remaining (included) after masking with
  \prog{ssu-mask}.  The column second from the right reports the size in 
  megabytes (Mb) of each full alignment file that is output by
  \prog{ssu-align}. These alignments include insert and consensus
  columns for all sequences. The rightmost column reports the size in
  Mb of the masked alignment output from \prog{ssu-mask}.  These
  alignments include a subset of the consensus columns and zero insert
  columns. See section~\ref{sec:background} for more information on
  consensus versus insert columns.}
\label{tbl:sizes}
\end{table}
%%%%%%%%%%%%%%%%%%%%%%%%%%%%%%%%%%%%
\begin{table}[hb]
\begin{center}
  \scriptsize
%  \begin{tabular}{llrr|rrrrr|rrr} 
  \begin{tabular}{llrr|rrrrr|rr} 
                &          &         &         &  & & & &                          & \multicolumn{2}{c}{parallel ssu-align} \\
                &          &         &average  & \multicolumn{5}{c|}{running time} & \multicolumn{2}{c}{statistics} \\ \cline {10-11} %& summed \\
                &          & \# of   &seq     & \multicolumn{5}{c|}{(hh:mm:ss)}   &                      \# of & wall  \\ \cline{5-9} %& time $\div$  \\ 
dataset         & models   & seqs    & length & ssu-prep & ssu-align & ssu-mask & ssu-draw & ssu-merge & procs & time  \\ \hline % & \# of procs \\ \hline
& & & & & & & & & \\
Ggenes-core     & default  &    4938 & 1437.0 &         - &   1:47:48 &   0:00:04 &   0:00:05 &         - &   1 &         - \\% &         - \\
Ggenes-full     & default  &  508194 & 1399.7 &   0:01:50 & 183:04:18 &   0:07:51 &   0:08:49 &   0:06:12 & 100 &   2:01:08 \\% &   1:49:51 \\
Silva-Ref       & default  &  460783 & 1437.1 &   0:01:18 & 175:07:23 &   0:13:13 &   0:15:25 &   0:09:03 & 100 &   1:58:50 \\% &   1:45:04 \\
Silva-Parc      & default  & 1246462 & 1017.0 &   0:02:28 & 315:20:54 &   1:07:07 &   1:05:40 &   0:24:27 & 100 &   3:47:06 \\% &   3:09:13 \\
RDP             & default  & 1237963 &  968.7 &   0:02:27 & 293:15:32 &   1:24:13 &   1:13:49 &   0:25:05 & 100 &   3:38:54 \\% &   2:55:57 \\
& & & & & & & & & \\ 
synthetic-full  & default  &     100 & 1603.7 &         - &   0:02:39 &   0:00:01 &   0:00:02 &         - &   1 &         - \\% &         - \\
synthetic-full  & default  &    1000 & 1606.9 &         - &   0:26:12 &   0:00:01 &   0:00:02 &         - &   1 &         - \\% &         - \\
synthetic-full  & default  &   10000 & 1607.0 &         - &   4:17:36 &   0:00:05 &   0:00:06 &         - &   1 &         - \\% &         - \\
synthetic-full  & default  &  100000 & 1606.9 &   0:00:51 &  44:56:37 &   0:00:57 &   0:01:14 &   0:01:08 & 100 &   0:30:22 \\% &   0:26:58 \\
synthetic-full  & default  & 1000000 & 1607.1 &   0:02:00 & 450:15:46 &   0:12:06 &   0:13:35 &   0:12:12 & 100 &   5:06:52 \\% &   4:30:09 \\
& & & & & & & & & \\
synthetic-v4    & default  &     100 &  239.2 &         - &   0:00:37 &   0:00:01 &   0:00:02 &         - &   1 &         - \\% &         - \\
synthetic-v4    & default  &    1000 &  239.2 &         - &   0:02:44 &   0:00:01 &   0:00:02 &         - &   1 &         - \\% &         - \\
synthetic-v4    & default  &   10000 &  239.2 &         - &   0:26:36 &   0:00:03 &   0:00:04 &         - &   1 &         - \\% &         - \\
synthetic-v4    & default  &  100000 &  239.2 &   0:01:02 &   4:37:51 &   0:00:07 &   0:00:23 &   0:00:27 & 100 &   0:03:17 \\% &   0:02:47 \\
synthetic-v4    & default  & 1000000 &  239.2 &   0:01:37 &  45:14:17 &   0:03:28 &   0:04:27 &   0:03:57 & 100 &   0:31:46 \\% &   0:27:09 \\
& & & & & & & & & \\
synthetic-v4    & v4       &     100 &  239.5 &         - &   0:00:06 &   0:00:01 &         - &         - &   1 &         - \\% &         - \\
synthetic-v4    & v4       &    1000 &  239.6 &         - &   0:00:34 &   0:00:01 &         - &         - &   1 &         - \\% &         - \\
synthetic-v4    & v4       &   10000 &  239.6 &         - &   0:05:31 &   0:00:01 &         - &         - &   1 &         - \\% &         - \\
synthetic-v4    & v4       &  100000 &  239.6 &   0:00:08 &   1:03:58 &   0:00:15 &         - &   0:00:14 & 100 &   0:00:55 \\% &   0:00:38 \\
synthetic-v4    & v4       & 1000000 &  239.6 &   0:00:23 &   9:42:32 &   0:02:60 &         - &   0:01:57 & 100 &   0:07:60 \\% &   0:05:50 \\ 
\end{tabular}
\end{center}
\caption{\textbf{Statistics and running times for various SSU
    datasets.} For each dataset, the number of and average length of all
  sequences is reported. The running time of each program:
  \prog{ssu-prep}, \prog{ssu-align}, \prog{ssu-mask}, \prog{ssu-draw},
and \prog{ssu-merge} is reported in the middle block. For datasets
with more than 10,000 sequences, \prog{ssu-prep} was used to create 100 parallel
\prog{ssu-align} jobs. For these datasets: the \emph{ssu-align} column
under \emph{running time} reports the summed time for all 100 jobs
minus the time required to merge the results; \emph{wall time} reports
the actual wall time required for the 100 jobs to run and be merged
together. All running times are given in \emph{hh:mm:ss} format;
\emph{hh}=hours, \emph{mm}=minutes, \emph{ss}=seconds. All
programs were run as single execution threads on 3.0 GHz Intel Xeon
processors.}
\label{tbl:ttimes}
\end{table}

%%%%%%%%%%%%%%%%%%%%%%%%%%%%%%%%%%%%
\begin{table}[hb]
\begin{center}
  \scriptsize
  \begin{tabular}{llrr|rrrr} 
                &          &         &        &  \multicolumn{4}{c}{ssu-align running time (hh:mm:ss)} \\ \cline{5-8}
                &          &         &average &           & stage 1      & stage 2      & both   \\ 
                &          &\# of    &sequence&           & (search)     & (alignment)  & stages \\ 
dataset         & models   &sequences& length & total     & per seq      & per seq      & per seq     \\ \hline
& & & & & & & \\
Ggenes-core     & default  &    4938 & 1437.0 &   1:47:48 &   0:00:00.64 &   0:00:00.67 &   0:00:01.31 \\
Ggenes-full     & default  &  508194 & 1399.7 & 183:04:18 &   0:00:00.63 &   0:00:00.67 &   0:00:01.30 \\
Silva-Ref       & default  &  460783 & 1437.1 & 175:07:23 &   0:00:00.66 &   0:00:00.71 &   0:00:01.37 \\
Silva-Parc      & default  & 1246462 & 1017.0 & 315:20:54 &   0:00:00.46 &   0:00:00.45 &   0:00:00.91 \\
RDP             & default  & 1237963 &  968.7 & 293:15:32 &   0:00:00.43 &   0:00:00.42 &   0:00:00.85 \\
& & & & & & & \\
%synthetic-full  & default  &     100 & 1603.7 &   0:02:39 &   0:00:00.78 &   0:00:00.78 &   0:00:01.58 \\
%synthetic-full  & default  &    1000 & 1606.9 &   0:26:12 &   0:00:00.79 &   0:00:00.78 &   0:00:01.57 \\
%synthetic-full  & default  &   10000 & 1607.0 &   4:17:36 &   0:00:00.75 &   0:00:00.79 &   0:00:01.55 \\
%synthetic-full  & default  &  100000 & 1606.9 &  44:56:37 &   0:00:00.79 &   0:00:00.83 &   0:00:01.62 \\
synthetic-full  & default  & 1000000 & 1607.1 & 450:15:46 &   0:00:00.78 &   0:00:00.84 &   0:00:01.62 \\
& & & & & & & \\
%synthetic-v4    & default  &     100 &  239.2 &   0:00:37 &   0:00:00.11 &   0:00:00.12 &   0:00:00.37 \\
%synthetic-v4    & default  &    1000 &  239.2 &   0:02:44 &   0:00:00.09 &   0:00:00.07 &   0:00:00.16 \\
%synthetic-v4    & default  &   10000 &  239.2 &   0:26:36 &   0:00:00.09 &   0:00:00.07 &   0:00:00.16 \\
%synthetic-v4    & default  &  100000 &  239.2 &   4:37:51 &   0:00:00.09 &   0:00:00.07 &   0:00:00.17 \\
synthetic-v4    & default  & 1000000 &  239.2 &  45:14:17 &   0:00:00.09 &   0:00:00.07 &   0:00:00.16 \\
& & & & & & & \\
%synthetic-v4    & v4       &     100 &  239.5 &   0:00:06 &   0:00:00.01 &   0:00:00.03 &   0:00:00.06 \\
%synthetic-v4    & v4       &    1000 &  239.6 &   0:00:34 &   0:00:00.01 &   0:00:00.02 &   0:00:00.03 \\
%synthetic-v4    & v4       &   10000 &  239.6 &   0:05:31 &   0:00:00.01 &   0:00:00.02 &   0:00:00.03 \\
%synthetic-v4    & v4       &  100000 &  239.6 &   1:03:58 &   0:00:00.01 &   0:00:00.02 &   0:00:00.04 \\
synthetic-v4    & v4       & 1000000 &  239.6 &   9:42:32 &   0:00:00.01 &   0:00:00.02 &   0:00:00.03 \\ 
\end{tabular}
\end{center}
\caption{\textbf{Statistics and \prog{ssu-align} per-sequence running
    times for various SSU datasets.} For each dataset, the number of and average length of all
  sequences is reported. The running time of \prog{ssu-align}
  is reported in the \emph{total} column. \prog{ssu-align} proceeds
  through two stages, the search stage first determines the domain of
  each sequence and the alignment stage structurally aligns each
  sequence to a domain specific model. The \emph{per seq} columns report
  the average time per sequence for the search stage, alignment stage,
  and for the entire program execution (\emph{both stages per seq}
  column). All running times are given in \emph{hh:mm:ss} format;
  \emph{hh}=hours, \emph{mm}=minutes, \emph{ss}=seconds. All
  programs were run as single execution threads on 3.0 GHz Intel Xeon
  processors.}
\label{tbl:ptimes}
\end{table}


Statistics are reported for real and synthetic datasets. The real
datasets were obtained exclusively from existing SSU databases:
GREENGENES \cite{DeSantis06a}, SILVA
\cite{Pruesse07} and RDP \cite{Cole09}. For GREENGENES,
two separate datasets were processed: the ``core set'', which is the
reference alignment used by that database (\emph{Ggenes-core}
dataset), and all the SSU sequences in the database
(\emph{Ggenes-full} dataset) after the March 23, 2010 update. Two
datasets from SILVA, \emph{Silva-Ref} and \emph{Silva-Parc}, were
processed. dataset The Ref set is a subset of the Parc
set that includes sequences that are generally longer and of higher
quality than the other sequences in Parc. Both sets of sequences are
from release 102 of the database. Finally, a single RDP dataset
was analyzed: the full set of sequences in release 10, update 20. The
\emph{Silva-Parc} and the \emph{RDP} datasets contain over 1 million
sequences.

The synthetic datasets were fabricated using the INFERNAL
package. Specifically, the \prog{cmemit} program was used with default
settings and the \prog{-n <N>} option to generate \prog{<N>}
sequences, where \prog{<N>} is reported in the \emph{number of seqs}
columns of the tables. For the \emph{synthetic-full} datasets, the
default three SSU-ALIGN SSU CMs were used to generate the
sequences, each contributing one third of the total sequences. For the
\emph{synthetic-v4} datasets, smaller, truncated versions of the default
archaeal and bacterial models were used to generate the sequences,
half each. These CMs only model the V4 hypervariable region of
SSU. They were constructed using the same \prog{ssu-build} commands
from the tutorial section (section
~\ref{sec:tutorial-build-v4}, page ~\pageref{sec:tutorial-build-v4}).

For all datasets the default models were used to align, mask and draw
sequences using \prog{ssu-align}, \prog{ssu-mask} and
\prog{ssu-draw}. These runs with default settings are summarized in the
rows of the tables that include \emph{default} in the \emph{models} column.
Additionally, for the \emph{synthetic-v4} dataset, the smaller V4
models were used for aligning and masking. Drawing was impossible for
these runs because non-default models were used. These runs correspond
to the rows in the tables marked \emph{v4} under \emph{model}.

For all datasets with more than 10,000 sequences \prog{ssu-prep} was
used to parallelize alignment. These datasets were split up into 100
separate jobs and run in parallel on a cluster. This is indicated by
the \emph{\# of procs} column in table~\ref{tbl:ttimes}.
For these datasets, additional columns in table~\ref{tbl:ttimes}
report the time required to run \prog{ssu-prep} and \prog{ssu-merge}
(which is automatically executed by the final \prog{ssu-align} job),
and the elapsed or \emph{wall} time required to perform all parallel
alignment jobs plus the final merge.

%~\ref{tbl:ptimes}. No such column exists in table ~\ref{tbl:sizes} because
%parallelization does not affect the sizes of the final output files.

As an example of a non-parallel run, the actual commands used to process the
GREENGENES core set dataset were as follows. The FASTA file
included the unaligned core set sequences is named \prog{ggcs-unaln.fa}. The
following three commands would align, mask and draw the core
set, respectively. All output files will be created in a directory called
\prog{ggcs}. Importantly, the three commands  must be run in succession
(masking cannot begin until alignment is complete). 

\user{ssu-align ggcs-unaln.fa ggcs}

\user{ssu-mask ggcs}

\user{ssu-draw ggcs}

As an example of a parallel run, the actual commands used to process the
GREENGENES full dataset were as follows. The FASTA file
included the unaligned core set sequences is named \prog{gg-unaln.fa}. The
following four commands would partition, align, mask and draw the full
dataset, respectively. All output files will be created in a directory
called \prog{gg}. Importantly, the four commands must be run in succession
(masking cannot begin until alignment is complete). 

\user{ssu-prep gg-unaln.fa gg 100 janelia-cluster-presuf.txt}

\user{sh gg.ssu-align.sh}

\user{ssu-mask gg}

\user{ssu-draw gg}

The \prog{ssu-prep} command above creates the
\prog{gg.ssu-align.sh} script that will submit 100 \prog{ssu-align}
jobs to the cluster here at Janelia. The Janelia-specific prefixes and
suffixes for the commands are in the file
\prog{janelia-cluster-presuf.txt} which is in the \prog{tutorial/}
subdirectory of SSU-ALIGN. See the tutorial
section~\ref{sec:tutorial-prep}, 
page~\pageref{sec:tutorial-prep} for more details on \prog{ssu-prep}.

The statistics in tables~\ref{tbl:sizes}, \ref{tbl:ttimes} and
\ref{tbl:ptimes} highlight some key points:

\begin{itemize}

\item Deep alignments are mostly composed of insert columns which
  dramatically inflate the size of the alignment files
  (table~\ref{tbl:sizes}). All insert columns will be removed
  automatically by \prog{ssu-mask} (see section~\ref{sec:background}
  for more discussion of insert versus consensus columns). 
  The full \emph{RDP} bacterial alignment of about 1.2 million
  sequences is 70 Gb and contains 29,736 total columns, 28,154 
  of which are inserts (about 95\%). The masked alignment of 1397
  columns is about 3.5 Gb. Note that insert columns will not be
  included in output alignments of \prog{ssu-align} when the
  \prog{--rfonly} option is used.

%\item Alignment is by far the slowest step in the pipeline
%  (table~\ref{tbl:ttimes}). Prep'ing, masking, drawing and merging are
%  all much faster than alignment, but only the \prog{ssu-align} step
%  can be parallelized. 

\item \prog{ssu-align} creates alignments of SSU sequences at a rate
  of roughly 1 sequence/second (table~\ref{tbl:ptimes}).

\item Short ($\sim$240 nt) V4 subsequences are aligned at about 10
  sequences per second with the default, full-length models (table~\ref{tbl:ptimes}).

\item Aligning short sequences with truncated models is significantly
  faster than with full models. V4 subsequences are aligned at about
  100 sequences per second with V4-specific models
  (table~\ref{tbl:ptimes}).

\item Aligning 1 million typical SSU sequences requires about 300
  CPU hours. This can be parallelized on a cluster of 100 processors
  in about 3.5 hours. Note that for about 25 of the final 30
  minutes only one processor is still running, this is the processor
  that is performing the merge of the 100 per-job alignments. 
  (table~\ref{tbl:ttimes}).

%\item Masking and drawing, though much faster than aligning, takes 
%  about an hour for a dataset of about 1 million sequences
% (table~\ref{tbl:ttimes}). The majority of this time is spent reading
%  in the huge alignment files. Using the \prog{ssu-align --rfonly}
%  option, since it reduces alignment file sizes, speeds up these steps
%  considerably (data not shown).
\end{itemize}


%%%%%%%%%%%%%%%%%%%%%%%%%%%%%%%%%%%%%%%%
% Notes on how these tables were created are in
% /groups/eddy/home/nawrockie/notebook/10_0322_ssualign_auto_merge/
% This work, as logged in the 00LOG in that dir, was done on May
% 24-27, 2010.
%%%%%%%%%%%%%%%%%%%%%%%%%%%%%%%%%%%%
\begin{table}[hb]
\begin{center}
  \scriptsize
  \begin{tabular}{lllrr|rrrr|rr} 
                &          &          &         &        &       &       &       &           &           & masked \\
                &          &          &         &average &total  & \# of & \# of & \# columns& alignment & alignment \\
                &          &          & \# of   &sequence&\# of  &insert &consensus&included & file      & file \\
dataset         & models   & domain   &sequences& length &columns&columns&columns&by mask    & size (Mb) & size (Mb) \\ \hline
& & & & & & & & & & \\ 
Ggenes-core     & default  & archaea  &     186 & 1407.5 &  1757 &   249 &  1508 &  1380 &      0.7 &      0.5 \\
                & default  & bacteria &    4752 & 1438.1 &  5319 &  3737 &  1582 &  1362 &     50.7 &     13.1 \\
& & & & & & & & & & \\ 
Ggenes-full     & default  & archaea  &    8407 & 1365.0 &  3151 &  1643 &  1508 &  1382 &     53.3 &     23.5 \\
                & default  & bacteria &  499787 & 1400.3 & 11900 & 10318 &  1582 &  1404 & 11910.9 &   1419.4 \\
& & & & & & & & & & \\ 
Silva-Ref       & default  & archaea  &   19260 & 1169.0 & 12107 & 10599 &  1508 &  1383 &    469.2 &     56.1 \\
                & default  & bacteria &  391158 & 1413.0 & 24540 & 22958 &  1582 &  1396 & 19254.6 &   1148.6 \\
                & default  & eukarya  &   50350 & 1727.3 & 20950 & 19069 &  1881 &  1418 &   2117.0 &    150.1 \\
& & & & & & & & & & \\ 
Silva-Parc      & default  & archaea  &   56113 &  862.1 & 14299 & 12791 &  1508 &  1383 &   1613.0 &    163.0 \\
                & default  & bacteria & 1062938 & 1016.4 & 27918 & 26336 &  1582 &  1391 & 59498.1 &   3105.0 \\
                & default  & eukarya  &  123454 & 1092.6 & 24247 & 22366 &  1881 &  1424 &   6003.9 &    368.6 \\
& & & & & & & & & & \\ 
RDP             & default  & archaea  &   55249 &  830.9 & 12724 & 11216 &  1508 &  1382 &   1411.7 &    158.4 \\
                & default  & bacteria & 1182624 &  975.1 & 29736 & 28154 &  1582 &  1397 & 70455.3 &   3426.5 \\ %\hline
%& & & & & & & & & & \\ \hline
%& & & & & & & & & & \\ %\hline
%synthetic-full  & default  & archaea  &      34 & 1479.2 &  1596 &    88 &  1508 &  1400 &      0.113 &      0.099 \\
%synthetic-full  & default  & bacteria &      33 & 1530.0 &  1660 &    78 &  1582 &  1480 &      0.114 &      0.102 \\
%synthetic-full  & default  & eukarya  &      33 & 1805.7 &  2343 &   462 &  1881 &  1550 &      0.161 &      0.107 \\
%& & & & & & & & & & \\ 
%%synthetic-full  & default  & archaea  &     334 & 1479.7 &  2040 &   532 &  1508 &  1424 &      1.381 &      0.968 \\
%%synthetic-full  & default  & bacteria &     333 & 1526.9 &  1872 &   290 &  1582 &  1476 &      1.265 &      1.001 \\
%%synthetic-full  & default  & eukarya  &     333 & 1814.4 &  3312 &  1431 &  1881 &  1551 &      2.226 &      1.050 \\
%%& & & & & & & & & & \\ 
%synthetic-full  & default  & archaea  &    3334 & 1480.2 &  3781 &  2273 &  1508 &  1423 &     25.366 &      9.638 \\
%synthetic-full  & default  & bacteria &    3333 & 1526.8 &  2844 &  1262 &  1582 &  1466 &     19.117 &      9.929 \\
%synthetic-full  & default  & eukarya  &    3333 & 1814.0 &  4827 &  2946 &  1881 &  1554 &     32.333 &     10.509 \\
%& & & & & & & & & & \\ 
%%synthetic-full  & default  & archaea  &   33334 & 1480.3 &  6312 &  4804 &  1508 &  1424 &    422.354 &     96.472 \\
%%synthetic-full  & default  & bacteria &   33333 & 1526.5 &  5217 &  3635 &  1582 &  1465 &    349.407 &     99.269 \\
%%synthetic-full  & default  & eukarya  &   33333 & 1813.9 &  7888 &  6007 &  1881 &  1552 &    527.411 &    105.002 \\
%%& & & & & & & & & & \\ 
%synthetic-full  & default  & archaea  &  333334 & 1480.4 &  8800 &  7292 &  1508 &  1424 &   5882.696 &    965.338 \\
%synthetic-full  & default  & bacteria &  333333 & 1526.6 &  7798 &  6216 &  1582 &  1465 &   5215.344 &    993.335 \\
%synthetic-full  & default  & eukarya  &  333333 & 1814.4 & 11342 &  9461 &  1881 &  1551 &   7577.349 &   1050.002 \\ %\hline
%& & & & & & & & & & \\ %\hline
%synthetic-v4    & default  & archaea  &      50 &  238.7 &  1519 &    11 &  1508 &   241 &      0.157 &      0.027 \\
%synthetic-v4    & default  & bacteria &      50 &  239.7 &  1591 &     9 &  1582 &   237 &      0.164 &      0.026 \\
%& & & & & & & & & & \\ 
%%synthetic-v4    & default  & archaea  &     500 &  238.9 &  1603 &    95 &  1508 &   241 &      1.626 &      0.262 \\
%%synthetic-v4    & default  & bacteria &     500 &  239.4 &  1614 &    32 &  1582 &   238 &      1.637 &      0.259 \\
%%& & & & & & & & & & \\ 
%synthetic-v4    & default  & archaea  &    5000 &  238.9 &  1927 &   419 &  1508 &   243 &     19.484 &      2.641 \\
%synthetic-v4    & default  & bacteria &    5000 &  239.5 &  1823 &   241 &  1582 &   239 &     18.444 &      2.601 \\
%& & & & & & & & & & \\ 
%%synthetic-v4    & default  & archaea  &   50000 &  238.9 &  2301 &   793 &  1508 &   246 &    232.305 &     26.801 \\
%%synthetic-v4    & default  & bacteria &   50000 &  239.5 &  2168 &   586 &  1582 &   240 &    219.004 &     26.201 \\
%%& & & & & & & & & & \\ 
%synthetic-v4    & default  & archaea  &  500000 &  238.9 &  2667 &  1159 &  1508 &   254 &   2690.005 &    277.001 \\
%synthetic-v4    & default  & bacteria &  500000 &  239.5 &  2563 &   981 &  1582 &   244 &   2586.005 &    267.001 \\ %\hline
%& & & & & & & & & & \\ %\hline
%synthetic-v4    & v4       & archaea  &      50 &  239.3 &   254 &    13 &   241 &   241 &      0.028 &      0.027 \\
%synthetic-v4    & v4       & bacteria   &      50 &  239.7 &   250 &     9 &   241 &   237 &      0.027 &      0.026 \\
%& & & & & & & & & & \\ 
%%synthetic-v4    & v4       & archaea  &     500 &  239.6 &   346 &   105 &   241 &   241 &      0.367 &      0.262 \\
%%synthetic-v4    & v4       & bacteria   &     500 &  239.7 &   278 &    37 &   241 &   238 &      0.299 &      0.259 \\
%%& & & & & & & & & & \\ 
%synthetic-v4    & v4       & archaea  &    5000 &  239.6 &   686 &   445 &   241 &   241 &      7.071 &      2.621 \\
%synthetic-v4    & v4       & bacteria   &    5000 &  239.7 &   490 &   249 &   241 &   238 &      5.111 &      2.591 \\
%%& & & & & & & & & & \\ 
%%synthetic-v4    & v4       & archaea  &   50000 &  239.6 &  1075 &   834 &   241 &   241 &    109.702 &     26.301 \\
%%synthetic-v4    & v4       & bacteria   &   50000 &  239.7 &   865 &   624 &   241 &   238 &     88.702 &     26.001 \\
%& & & & & & & & & & \\ 
%synthetic-v4    & v4       & archaea   &  500000 &  239.6 &  1469 &  1228 &   241 &   241 &   1492.003 &    264.001 \\
%synthetic-v4    & v4       & bacteria    &  500000 &  239.7 &  1283 &  1042 &   241 &   238 &   1306.003 &    261.001 \\
\end{tabular}
\end{center}
\caption{\textbf{Statistics and resulting output file sizes for
    various SSU datasets.} For each dataset, the number
  of and average length of sequences for each domain, as classified by
  \prog{ssu-align}, is reported. For each dataset and domain an
  alignment is created. The four columns in the middle block report
  the total number of columns, the
  number of insert and consensus columns (consensus column counts are
  invariant per domain across all datasets), and the number of
  consensus columns remaining (included) after masking with
  \prog{ssu-mask}.  The column second from the right reports the size in 
  megabytes (Mb) of each full alignment file that is output by
  \prog{ssu-align}. These alignments include insert and consensus
  columns for all sequences. The rightmost column reports the size in
  Mb of the masked alignment output from \prog{ssu-mask}.  These
  alignments include a subset of the consensus columns and zero insert
  columns. See section~\ref{sec:background} for more information on
  consensus versus insert columns.}
\label{tbl:sizes}
\end{table}
%%%%%%%%%%%%%%%%%%%%%%%%%%%%%%%%%%%%
\begin{table}[hb]
\begin{center}
  \scriptsize
%  \begin{tabular}{llrr|rrrrr|rrr} 
  \begin{tabular}{llrr|rrrrr|rr} 
                &          &         &         &  & & & &                          & \multicolumn{2}{c}{parallel ssu-align} \\
                &          &         &average  & \multicolumn{5}{c|}{running time} & \multicolumn{2}{c}{statistics} \\ \cline {10-11} %& summed \\
                &          & \# of   &seq     & \multicolumn{5}{c|}{(hh:mm:ss)}   &                      \# of & wall  \\ \cline{5-9} %& time $\div$  \\ 
dataset         & models   & seqs    & length & ssu-prep & ssu-align & ssu-mask & ssu-draw & ssu-merge & procs & time  \\ \hline % & \# of procs \\ \hline
& & & & & & & & & \\
Ggenes-core     & default  &    4938 & 1437.0 &         - &   1:47:48 &   0:00:04 &   0:00:05 &         - &   1 &         - \\% &         - \\
Ggenes-full     & default  &  508194 & 1399.7 &   0:01:50 & 183:04:18 &   0:07:51 &   0:08:49 &   0:06:12 & 100 &   2:01:08 \\% &   1:49:51 \\
Silva-Ref       & default  &  460783 & 1437.1 &   0:01:18 & 175:07:23 &   0:13:13 &   0:15:25 &   0:09:03 & 100 &   1:58:50 \\% &   1:45:04 \\
Silva-Parc      & default  & 1246462 & 1017.0 &   0:02:28 & 315:20:54 &   1:07:07 &   1:05:40 &   0:24:27 & 100 &   3:47:06 \\% &   3:09:13 \\
RDP             & default  & 1237963 &  968.7 &   0:02:27 & 293:15:32 &   1:24:13 &   1:13:49 &   0:25:05 & 100 &   3:38:54 \\% &   2:55:57 \\
& & & & & & & & & \\ 
synthetic-full  & default  &     100 & 1603.7 &         - &   0:02:39 &   0:00:01 &   0:00:02 &         - &   1 &         - \\% &         - \\
synthetic-full  & default  &    1000 & 1606.9 &         - &   0:26:12 &   0:00:01 &   0:00:02 &         - &   1 &         - \\% &         - \\
synthetic-full  & default  &   10000 & 1607.0 &         - &   4:17:36 &   0:00:05 &   0:00:06 &         - &   1 &         - \\% &         - \\
synthetic-full  & default  &  100000 & 1606.9 &   0:00:51 &  44:56:37 &   0:00:57 &   0:01:14 &   0:01:08 & 100 &   0:30:22 \\% &   0:26:58 \\
synthetic-full  & default  & 1000000 & 1607.1 &   0:02:00 & 450:15:46 &   0:12:06 &   0:13:35 &   0:12:12 & 100 &   5:06:52 \\% &   4:30:09 \\
& & & & & & & & & \\
synthetic-v4    & default  &     100 &  239.2 &         - &   0:00:37 &   0:00:01 &   0:00:02 &         - &   1 &         - \\% &         - \\
synthetic-v4    & default  &    1000 &  239.2 &         - &   0:02:44 &   0:00:01 &   0:00:02 &         - &   1 &         - \\% &         - \\
synthetic-v4    & default  &   10000 &  239.2 &         - &   0:26:36 &   0:00:03 &   0:00:04 &         - &   1 &         - \\% &         - \\
synthetic-v4    & default  &  100000 &  239.2 &   0:01:02 &   4:37:51 &   0:00:07 &   0:00:23 &   0:00:27 & 100 &   0:03:17 \\% &   0:02:47 \\
synthetic-v4    & default  & 1000000 &  239.2 &   0:01:37 &  45:14:17 &   0:03:28 &   0:04:27 &   0:03:57 & 100 &   0:31:46 \\% &   0:27:09 \\
& & & & & & & & & \\
synthetic-v4    & v4       &     100 &  239.5 &         - &   0:00:06 &   0:00:01 &         - &         - &   1 &         - \\% &         - \\
synthetic-v4    & v4       &    1000 &  239.6 &         - &   0:00:34 &   0:00:01 &         - &         - &   1 &         - \\% &         - \\
synthetic-v4    & v4       &   10000 &  239.6 &         - &   0:05:31 &   0:00:01 &         - &         - &   1 &         - \\% &         - \\
synthetic-v4    & v4       &  100000 &  239.6 &   0:00:08 &   1:03:58 &   0:00:15 &         - &   0:00:14 & 100 &   0:00:55 \\% &   0:00:38 \\
synthetic-v4    & v4       & 1000000 &  239.6 &   0:00:23 &   9:42:32 &   0:02:60 &         - &   0:01:57 & 100 &   0:07:60 \\% &   0:05:50 \\ 
\end{tabular}
\end{center}
\caption{\textbf{Statistics and running times for various SSU
    datasets.} For each dataset, the number of and average length of all
  sequences is reported. The running time of each program:
  \prog{ssu-prep}, \prog{ssu-align}, \prog{ssu-mask}, \prog{ssu-draw},
and \prog{ssu-merge} is reported in the middle block. For datasets
with more than 10,000 sequences, \prog{ssu-prep} was used to create 100 parallel
\prog{ssu-align} jobs. For these datasets: the \emph{ssu-align} column
under \emph{running time} reports the summed time for all 100 jobs
minus the time required to merge the results; \emph{wall time} reports
the actual wall time required for the 100 jobs to run and be merged
together. All running times are given in \emph{hh:mm:ss} format;
\emph{hh}=hours, \emph{mm}=minutes, \emph{ss}=seconds. All
programs were run as single execution threads on 3.0 GHz Intel Xeon
processors.}
\label{tbl:ttimes}
\end{table}

%%%%%%%%%%%%%%%%%%%%%%%%%%%%%%%%%%%%
\begin{table}[hb]
\begin{center}
  \scriptsize
  \begin{tabular}{llrr|rrrr} 
                &          &         &        &  \multicolumn{4}{c}{ssu-align running time (hh:mm:ss)} \\ \cline{5-8}
                &          &         &average &           & stage 1      & stage 2      & both   \\ 
                &          &\# of    &sequence&           & (search)     & (alignment)  & stages \\ 
dataset         & models   &sequences& length & total     & per seq      & per seq      & per seq     \\ \hline
& & & & & & & \\
Ggenes-core     & default  &    4938 & 1437.0 &   1:47:48 &   0:00:00.64 &   0:00:00.67 &   0:00:01.31 \\
Ggenes-full     & default  &  508194 & 1399.7 & 183:04:18 &   0:00:00.63 &   0:00:00.67 &   0:00:01.30 \\
Silva-Ref       & default  &  460783 & 1437.1 & 175:07:23 &   0:00:00.66 &   0:00:00.71 &   0:00:01.37 \\
Silva-Parc      & default  & 1246462 & 1017.0 & 315:20:54 &   0:00:00.46 &   0:00:00.45 &   0:00:00.91 \\
RDP             & default  & 1237963 &  968.7 & 293:15:32 &   0:00:00.43 &   0:00:00.42 &   0:00:00.85 \\
& & & & & & & \\
%synthetic-full  & default  &     100 & 1603.7 &   0:02:39 &   0:00:00.78 &   0:00:00.78 &   0:00:01.58 \\
%synthetic-full  & default  &    1000 & 1606.9 &   0:26:12 &   0:00:00.79 &   0:00:00.78 &   0:00:01.57 \\
%synthetic-full  & default  &   10000 & 1607.0 &   4:17:36 &   0:00:00.75 &   0:00:00.79 &   0:00:01.55 \\
%synthetic-full  & default  &  100000 & 1606.9 &  44:56:37 &   0:00:00.79 &   0:00:00.83 &   0:00:01.62 \\
synthetic-full  & default  & 1000000 & 1607.1 & 450:15:46 &   0:00:00.78 &   0:00:00.84 &   0:00:01.62 \\
& & & & & & & \\
%synthetic-v4    & default  &     100 &  239.2 &   0:00:37 &   0:00:00.11 &   0:00:00.12 &   0:00:00.37 \\
%synthetic-v4    & default  &    1000 &  239.2 &   0:02:44 &   0:00:00.09 &   0:00:00.07 &   0:00:00.16 \\
%synthetic-v4    & default  &   10000 &  239.2 &   0:26:36 &   0:00:00.09 &   0:00:00.07 &   0:00:00.16 \\
%synthetic-v4    & default  &  100000 &  239.2 &   4:37:51 &   0:00:00.09 &   0:00:00.07 &   0:00:00.17 \\
synthetic-v4    & default  & 1000000 &  239.2 &  45:14:17 &   0:00:00.09 &   0:00:00.07 &   0:00:00.16 \\
& & & & & & & \\
%synthetic-v4    & v4       &     100 &  239.5 &   0:00:06 &   0:00:00.01 &   0:00:00.03 &   0:00:00.06 \\
%synthetic-v4    & v4       &    1000 &  239.6 &   0:00:34 &   0:00:00.01 &   0:00:00.02 &   0:00:00.03 \\
%synthetic-v4    & v4       &   10000 &  239.6 &   0:05:31 &   0:00:00.01 &   0:00:00.02 &   0:00:00.03 \\
%synthetic-v4    & v4       &  100000 &  239.6 &   1:03:58 &   0:00:00.01 &   0:00:00.02 &   0:00:00.04 \\
synthetic-v4    & v4       & 1000000 &  239.6 &   9:42:32 &   0:00:00.01 &   0:00:00.02 &   0:00:00.03 \\ 
\end{tabular}
\end{center}
\caption{\textbf{Statistics and \prog{ssu-align} per-sequence running
    times for various SSU datasets.} For each dataset, the number of and average length of all
  sequences is reported. The running time of \prog{ssu-align}
  is reported in the \emph{total} column. \prog{ssu-align} proceeds
  through two stages, the search stage first determines the domain of
  each sequence and the alignment stage structurally aligns each
  sequence to a domain specific model. The \emph{per seq} columns report
  the average time per sequence for the search stage, alignment stage,
  and for the entire program execution (\emph{both stages per seq}
  column). All running times are given in \emph{hh:mm:ss} format;
  \emph{hh}=hours, \emph{mm}=minutes, \emph{ss}=seconds. All
  programs were run as single execution threads on 3.0 GHz Intel Xeon
  processors.}
\label{tbl:ptimes}
\end{table}

